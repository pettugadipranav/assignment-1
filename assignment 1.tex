\documentclass[15PT]{article}
\usepackage[utf8]{inputenc}
\usepackage{amsmath}
\usepackage{multicol}
\setlength{\columnsep}{1cm}
\setlength{\columnseprule}{1pt}
 \usepackage{polynom}
\usepackage{listings}
\lstset{
frame=single, 
breaklines=true,
columns=fullflexible
}
\begin{document}
\title{AI1110 ASSIGNMENT 1}
\author{Pranav Pettugadi}
\date{30 March 2022}
\maketitle

\section*{Q 10(a)}
  The sum of the ages of vivek and his younger brother amit is 47 years.The product of their ages in years is 550.Find their ages.
  \begin{multicols*}{2}
  
  \section*{solution}
  Let the age of vivek be 'a' and the age of amit be 'b'.$ a,b \in N$
  
  Given
  \begin{equation}
  a>b 
  \end{equation}
  
    
  \begin{equation}
            a+b=47
 \end{equation}
 \begin{equation}
     ab=550
 \end{equation}
 
 from(3)
 \begin{align*}
 b&=\frac{550}{a}\\
 \end{align*}
 
 substituting (4) in (2)
 \begin{align*}
    \Rightarrow & a+\frac{550}{a}=47\\
    \Rightarrow & a^2-47a+550=0\\
    \Rightarrow & a^2-(25+22)a+(25\times22)=0\\
    \Rightarrow & a=25 \hspace{8pt}or\hspace{8pt} a=22
\end{align*}
\begin{center}
    

  \hspace{15pt} according to (1)\\
   \hspace{8pt}discarding $a=22$ as if $a=22 $ then $b=25$ \\
\end{center}
 
 \hspace{15pt} so a=25 and b=22 \\
The age of vivek is 25 years and age of amit is 22 years
\end{multicols*}
 \end{document}
