\let\negmedspace\undefined
\let\negthickspace\undefined
\documentclass[journal,12pt,twocolumn]{IEEEtran}
\usepackage[utf8]{inputenc}
\usepackage{amsmath}
\usepackage{amssymb}
 \usepackage{polynom}
\usepackage{listings}
\lstset{
frame=single, 
breaklines=true,
columns=fullflexible
}
\begin{document}
\newcommand{\question}{\noindent \textbf{Question: }}	
\newcommand{\solution}{\noindent \textbf{Solution: }}
\title{AI1110 ASSIGNMENT 1}
\author{Pranav Pettugadi}
\date{30 March 2022}
\maketitle

\question : 10(a) :
  The sum of the ages of vivek and his younger brother amit is 47 years.The product of their ages in years is 550.Find their ages.
  
  \solution
  Let the age of vivek be 'a' and the age of amit be 'b'.$ a,b \in N$
  
  Given
  \begin{equation}
\label{maineq}
a>b
  \end{equation}
 \begin{equation}
\label{eq1}
     a+b=47
 \end{equation}
\begin{equation}
 \label{eq2}
 ab=550
\end{equation}
 
  from(3)
 \begin{equation}
b=\frac{550}{a}     
 \end{equation}
 
 substituting (4) in (2)
 \begin{align*}
    \implies & a+\frac{550}{a}=47\\
    \implies & a^2-47a+550=0\\
    \implies & a^2-(25+22)a+(25\times22)=0\\
    \implies & a=25 \hspace{8pt}or\hspace{8pt} a=22
\end{align*}
\begin{center}
     according to (1)\\
   discarding $a=22$ as if $a=22 $ then $b=25$ \\
\end{center}

 $ \therefore $ the value of $a=25$ and value of $b=22$ \\
The age of vivek is 25 years and age of amit is 22 years

 \end{document}
